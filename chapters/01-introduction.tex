\chapter{Introduction de l'équipe}

Notre équipe est composée de 8 personnes:

\paragraph{Joey Brynckman}
\paragraph{Andrea Dal Molin}
\paragraph{Thibault Havet}
\paragraph{Flo Raeymaeckers} (DevOps \& Responsable Équipe)

Je me suis occupée de l'infrastructure logicielle du déploiement de la solution. Plus précisément, j'ai mis en place un cluster de serveurs afin que l'ensemble des services que l'on voulait déployer sur le serveur puisse profiter d'un load balancing et d'une virtualisation par container afin d'avoir une scalabilité horizontale des services en fonction de la demande.

De plus, j'ai géré l'équipe. J'ai travaillé sur la mise en commun des ressources (réunions, outils de suivi de projet...), de ce rapport, et enfin de la présentation que vous allez pouvoir assister. Je vais passer en revue dans ce document les différentes difficultés par lesquelles je suis passée et les éventuelles solutions que j'ai apporté.

\paragraph{Arnaud Rase}
\paragraph{William Tea}
\paragraph{Lionel Thys}
\paragraph{Alexandre Villance}

\chapter{Étude et mise en théorie de l'infrastructure}

Dans ce chapitre, nous allons passer en revue l'étude que nous avons effectué sur l'énoncé du projet transmit pas nos professeurs. Le projet part d'un énoncé basé sur une version "simplifiée" de 3D secure, ou du moins plus ou moins cousine: nous avons un marchand qui souhaite pouvoir proposer à ses clients la vente de ses services/bien en ligne. Afin que le consommateur puisse payer, le marchand a besoin d'une plateforme de paiement en ligne. Notre rôle est d'implémenter cette plateforme de payement au niveau de la banque du consommateur, qui va alors se synchroniser avec la banque du marchand pour valider la transaction.

\section{Schéma de l'infrastructure}
