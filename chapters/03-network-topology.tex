\section{Topologie Réseau}

\begin{figure}[H]
    \centering
    \includegraphics[width=\textwidth]{./img/Projet réseau.png}
    \caption{Topologie du réseau}
    \label{fig:arnaud-topo}
\end{figure}

Pour la partie routage du réseau, le protocole BGP a été mis en place afin de communiquer
entre les différents ISPs et systèmes autonomes (AS). L'utilisation de 2 ISPs différent permet
l'accès à internet et la connectivité avec d'autres réseaux externes. BGP quant à lui prend en
charge le routage dans un environnement où plusieurs AS interagissent. La double relation
entre les routeurs R2 et R3 sont de l'IBGP alors que celles entre les ISPs, l'internet et les
routeurs 2 et 3 sont de type EBGP. Afin de faciliter la communication entre les différents liens,
l'interface de loopback a été utilisée pour donner une interface de « référence » virtuelle qui
ne changera pour s'occuper de ces communications. Il a également été imposé que les deux
ISPs ne puissent pas communiquer entre eux via le réseau privé, celui-ci ne doit pas servir de
zone de transit.

Une fois rentré dans la partie « privée » du réseau, c'est le protocole EIGRP qui prend le relai
afin de rediriger les paquets venant de l'internet vers routeur 1 qui aura la charge de PAT/NAT.
Contrairement au BGP utilisé pour communiquer entre plusieurs AS, EIGRP se concentre sur
le routage au sein d'un même AS. Une fois la traduction terminée, R1 va pouvoir amener les
paquets vers le serveur qui sera utilisé.

Pour accéder au serveur, le paquet passera par une partie switchée qui a le spanning-tree
protocol de configuré. Celui-ci permet d'empêcher les boucles potentielles tout en conservant
une certaine redondance. Avec cela les potentiels problèmes de congestions liés aux boucles
n'auront pas lieu.

L'utilisation conjointe de VLAN et de HSRP permet de gérer d'une façon plus fluide et plus
adaptée les différentes requêtes de chaque serveur.

Le VLAN permet d'avoir une sécurité améliorée en segmentant le réseau en groupes logiques
distincts. Cette segmentation profite également au niveau des performances en répartissant
les charges pour une meilleure gestion des ressources.

Le HSRP permet lui une meilleure disponibilité du réseau d'une façon un peu similaire au
spanning-tree. L'utilisation simultanée de ces deux protocoles permet une gestion efficace des
ressources, une redondance et un rétablissement automatique.

Dans le cadre de ce projet, il a été convenu d'utiliser la version d'essai de Cisco Modeling Labs.
Malheureusement celle-ci ne vient pas sans restriction, la connexion au réseau externe via un
« external connector » ne fonctionne pas empêchant d'intégrer la topologie montrée
précédemment au reste du projet. Le « switch virtuel » présent dans la topologie est là pour
représenter à quoi aurait dû ressembler la topologie si la connexion avait été possible.
